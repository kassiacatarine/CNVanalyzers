% RESUMO--------------------------------------------------------------------------------

\begin{resumo}[RESUMO]
\begin{SingleSpacing}

% Não altere esta seção do texto--------------------------------------------------------
\imprimirautorcitacao. \imprimirtitulo. \imprimirdata. \pageref {LastPage} f. \imprimirprojeto\ – \imprimirprograma, \imprimirinstituicao. \imprimirlocal, \imprimirdata.\\
%---------------------------------------------------------------------------------------

% Definir -- tema
A variação no número de cópia (Copy Number Variation/CNV) é uma alteração estrutural caracterizada pela inserção e/ou deleção de uma determinada região do genoma humano.
% Importância -- tema
A análise de CNVs é um fator relevante cientificamente, pois as variações ocorridas no genoma podem contribuir para as características genéticas e o surgimento de doenças mendelianas.
% Descrever como elas podem ser vistas -- tema
As tecnologias de sequenciamento do DNA, são capazes de extrair as informações necessárias para que a análise e identificação de CNVs possa ser realizada. \\
% Descrever como elas podem ser detectadas -- tema
% Descrever do que se trata o trabalho -- Objetivo
% Como ele irá ser feito -- método
% O que foi feito -- resultado
% O que se conclui a partir disso -- conclusão


% Antigo
% A análise de variações no número de cópias (CNV) é uma das principais fontes de estudo acerca de doenças e características genéticas. As tecnologias de sequenciamento desenvolvidas permitiu a aplicação de métodos estatísticos, como a detecção de pontos de mudanças (CPD) em seus dados. A técnica de CPD pode determinar variações ocorridas em uma série temporal, identificando as suas localidades, ao aplicar cálculos de acordo com o algoritmo imposto. Propõe-se nesse projeto buscar fornecer uma nova ferramenta de detecção de CNVs com adaptações de vários métodos de CPD, para a leitura de linhas celulares, a partir de dados sequenciados do exoma. \\

\textbf{Palavras-chave}: Variação no número de cópia (Copy Number Variation/CNV). Change Point Detection (CPD). Análise de sequenciamento completo do exoma (WES).

\end{SingleSpacing}
\end{resumo}

% OBSERVAÇÕES---------------------------------------------------------------------------
% Altere o texto inserindo o Resumo do seu trabalho.
% Escolha de 3 a 5 palavras ou termos que descrevam bem o seu trabalho 
