% INTRODUÇÃO-------------------------------------------------------------------

\chapter{INTRODUÇÃO}
\label{chap:introducao}

% Contextualizar do estudo de CNV

O genoma é uma grande fonte de informações acerca das nossas características e hereditariedade \cite{Correa2008}. O estudo das pequenas frações de regiões codificadoras de proteína do genoma (exoma) é bastante importante para a análise de doenças de origem genética, pois nele é contido cerca de 85\% de variantes conhecidos ligados a doenças \cite{Chong2015}.

O exoma pode ser extraído por meio de um sequenciamento, onde seus variantes ficam visíveis para aplicação de algoritmos de análises e estudo de especialistas (referencia). O avanço tecnológico para detecção de padrões e anomalias nos dados gerados aparti do exoma tem-se evoluído ao longo do tempo, permitindo a observação das variações dos números de cópias (CNV) de cromossomos no DNA (referencia).

A partir do sequenciamento do exoma é possível realizar uma distribuição dos dados mapeados em um gráfico, assim originando uma série temporal, ou seja, um conjunto de dados distribuídos ao longo de um determinado tempo (referencia). Desta forma, aumentando viabilidade de se analisar o CNV em determinadas posições.

O Copy Number Variation (CNV) é observado quando ocorre alguma inserção ou deleção no cromossomo, fazendo com que ele deixe de ser diploide (referencia). Com a distribuição do dataset gerado pelo sequenciamento do exoma em um gráfico, os CNV ficam visíveis para serem estudados, mas a indentificação do ponto onde ocorre algum tipo de CNV não é facil de ser apontado e ou classificado, assim dificultando o apontamento de doenças genéticas.

O reconhecimento da mudança ou ponto de mudança com uma maior precisão é um dos grandes desafios relacionado a análise estatística (referencia). O CP é fundamental para dar continuidade em pesquisas relacionadas a doenças ocorridas por algum CNV (referencia).

O \textit{Change Point} é pouco perceptivel a olho nu, dificultando o seu reconhecimento com uma maior precisão. Embora haja algoritmos que o detectam, como por exemplo \textit{Circular Binary Segmentation} ele não possue uma performace adequada para analise das segmentações mais recentes, pois elas contém uma quantidade imensa de marcadores assim aumentando o número de calculos necessários para indentificar a mudança em um determinado ponto por causa do uso de permutação em seu algoritmo.

No meio estatistico, existe a utilização e implementação de diversos calculos para solucionar tal questão, como  

Diante das questões expostas, a exploração de novas algoritmos capaz de demonstrar uma melhor apresentação para detectar o CP vem sendo estudada, tanto no meio da estastistica quanto no da bioinformatica (referencia). Mas com a evolução e melhora das tecnologias, temos cada vez mais nitidez nas extração dos dados do exoma, assim necissitando de algoritmos e tecnicas mais potentes para inspeção.

\section{JUSTIFICATIVA}

\section{OBJETIVOS}

\subsection{Objetivo Geral}

\subsection{Objetivo Especifico}

\section{ORGANIZAÇÃO DO TEXTO}