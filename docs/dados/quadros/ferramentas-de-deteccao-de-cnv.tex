% ##### define colors
\definecolor{color000000}{RGB}{0,0,0}
% ##### FIM define colors

% ######## init table ########
\begin{table}[h]
 \centering
% distancia entre a linha e o texto
 {\renewcommand\arraystretch{1.25}
 \caption{Ferramentas de detecção de CNVs usando a abordagem de profundidade em leitura\label{qua:ferramentas-de-deteccao-de-cnv}}
 \begin{adjustbox}{max width=\textwidth}
\begin{tabular}{ l l l l l }
  \cline{1-1}\cline{2-2}\cline{3-3}\cline{4-4}\cline{5-5}  
    \multicolumn{1}{|c|}{\textbf{\textcolor{color000000}{Ferramenta}}} &
    \multicolumn{1}{c|}{\textbf{\textcolor{color000000}{Versão}}} &
    \multicolumn{1}{c|}{\textbf{\textcolor{color000000}{Linguagem}}} &
    \multicolumn{1}{c|}{\textbf{\textcolor{color000000}{Método}}} &
    \multicolumn{1}{c|}{\textbf{Referência}}
  \\  
  \cline{1-1}\cline{2-2}\cline{3-3}\cline{4-4}\cline{5-5}  
    \multicolumn{1}{|c|}{DNAcopy} &
    \multicolumn{1}{c|}{1.58.0} &
    \multicolumn{1}{c|}{R} &
    \multicolumn{1}{c|}{Segmentação binária circular (CBS)} &
    \multicolumn{1}{c|}{\cite{Olshen2004}}
  \\  
  \cline{1-1}\cline{2-2}\cline{3-3}\cline{4-4}\cline{5-5}  
    \multicolumn{1}{|c|}{fastseg} &
    \multicolumn{1}{c|}{1.30.0} &
    \multicolumn{1}{c|}{R} &
    \multicolumn{1}{c|}{Estrutura Bayesiana} &
    \multicolumn{1}{c|}{\cite{Baldi2001}}
  \\  
  \cline{1-1}\cline{2-2}\cline{3-3}\cline{4-4}\cline{5-5}  
    \multicolumn{1}{|c|}{iSeg} &
    \multicolumn{1}{c|}{1.3.2} &
    \multicolumn{1}{c|}{C++} &
    \multicolumn{1}{c|}{\textit{T-}testes simples para calcular \textit{p-}valores} &
    \multicolumn{1}{c|}{\cite{Girimurugan2018}}
  \\  
  \cline{1-1}\cline{2-2}\cline{3-3}\cline{4-4}\cline{5-5}  
    \multicolumn{1}{|c|}{CGHSeg} &
    \multicolumn{1}{c|}{1.0.2} &
    \multicolumn{1}{c|}{R} &
    \multicolumn{1}{c|}{Modelos lineares misto} &
    \multicolumn{1}{c|}{\cite{Picard2011}}
  \\
  \cline{1-1}\cline{2-2}\cline{3-3}\cline{4-4}\cline{5-5}  
    \multicolumn{1}{|c|}{ExomeDepth} &
    \multicolumn{1}{c|}{1.1.10} &
    \multicolumn{1}{c|}{R} &
    \multicolumn{1}{c|}{Modelo beta-binomial para ajuste de profundidade} &
    \multicolumn{1}{c|}{\cite{Plagnol2012}}
  \\ 
  \hline

 \end{tabular}
 \end{adjustbox}}
\end{table}